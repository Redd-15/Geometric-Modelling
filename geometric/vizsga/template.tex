\documentclass[a4paper]{article}

% Set margins
\usepackage[hmargin=2.5cm, vmargin=3cm]{geometry}

\frenchspacing

% Language packages
\usepackage[utf8]{inputenc}
\usepackage[T1]{fontenc}
\usepackage[magyar]{babel}

% AMS
\usepackage{amssymb,amsmath}

\usepackage{xcolor}

\begin{document}

\pagestyle{empty}

Miskolci Egyetem, Matematikai Intézet

\hskip 10cm Név:

\medskip

\hskip 10cm Neptun-kód:

\begin{center}
   \large \textbf{Vizsga zárthelyi dolgozat\\
   GEOMETRIAI MODELLEZÉS (GEAGT232-ML) c. tárgyból}
\end{center}

\bigskip

\noindent \textbf{1. Feladat}
Ismertesse az $\textbf{r}(t), t \in [a, b]$ térgörbe kísérőtriéderét! Nevezze meg ennek vektorait, síkjait! \\
\textit{(4 pont)}

\bigskip

\noindent \textbf{2. Feladat}
Ismertesse az Overhauser spline-t!
Írja le, hogy mi az ami adott, mit keresünk! Részletezze a számításokat és szemléltesse ábrán!
\textit{(6 pont)}

\bigskip

\noindent \textbf{3. Feladat}
Görbék paraméterezése -- Mutassa be, hogy interpolációs görbék esetében melyek a jellemző paramétermegadási módok!
\textit{(4 pont)}

\bigskip

\noindent \textbf{4. Feladat}
Írja fel a Bernstein polinomot, majd ennek segítségével a Bézier görbét!
\textit{(6 pont)}

\bigskip

\noindent \textbf{5. Feladat}
Mutassa be a felületek paraméteres megadási módját! Térjen ki a felületi normálvektorra és az érintősíkra!
\textit{(4 pont)}

\bigskip

\noindent \textbf{Képfeldolgozás} Ismertesse (2 dimenziós képek esetén) a medián szűrő működését!

\bigskip

\noindent \textit{Ponthatárok: 0-11 elégtelen, 12-15 elégséges, 16-18 közepes, 19-21 jó, 22-24 jeles}

\newpage


Miskolci Egyetem, Matematikai Intézet

\hskip 10cm Név:

\medskip

\hskip 10cm Neptun-kód:

\begin{center}
	\large \textbf{Vizsga zárthelyi dolgozat\\
		GEOMETRIAI MODELLEZÉS (GEAGT232-ML) c. tárgyból}
\end{center}

\bigskip

\noindent \textbf{1. Feladat}
Ismertesse az $\textbf{r}(t), t \in [a, b]$ térgörbe kísérőtriéderét! Nevezze meg ennek vektorait, síkjait! \\
\textit{(4 pont)}

\bigskip

\noindent \textbf{2. Feladat}
Ismertesse a Hermit ívet!
Írja le, hogy mi az ami adott, mit keresünk! Részletezze a számításokat és szemléltesse ábrán!
\textit{(6 pont)}

\bigskip

\noindent \textbf{3. Feladat}
Görbék paraméterezése -- Mutassa be, hogy interpolációs görbék esetében melyek a jellemző paramétermegadási módok!
\textit{(4 pont)}

\bigskip

\noindent \textbf{4. Feladat}
Mutassa be a de Casteljau-algoritmust, és vele a Bézier görbe előállítását!
\textit{(6 pont)}

\bigskip

\noindent \textbf{5. Feladat}
Mutassa be a felületek paraméteres megadási módját! Térjen ki a felületi normálvektorra és az érintősíkra!
\textit{(4 pont)}

\bigskip

\noindent \textbf{Képfeldolgozás} Ismertesse a hisztogram széthúzás műveletét szürkeárnyalatos képek esetén!

\bigskip

\noindent \textit{Ponthatárok: 0-11 elégtelen, 12-15 elégséges, 16-18 közepes, 19-21 jó, 22-24 jeles}

\end{document}
